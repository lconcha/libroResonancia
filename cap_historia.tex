
La técnica de  resonancia magnética y la aplicación clínica que tiene hoy en día es resultado de años de investigación de diferentes grupos de investigación, y es el ejemplo de cómo los esfuerzos de la investigación básica pueden llegar a tener después de un tiempo un rango alto de aplicabilidad.
La resonancia magnética surgió como resultado de la investigación de la naturaleza de la materia, en este campo, un hallazgo importante fue el de Pauli, físico austriaco que en los años veinte propuso el cuarto número cuántico: el spin del electrón y la existencia del spin nuclear, demostró que muchas de las características de la estructura hiperfina de los espectros atómicos se podían explicar si los núcleos atómicos presentaban spin, que es una partícula giratoria que genera un campo magnético y un momento magnético asociado y actúa como una pequeña barra magnética con polos positivo y negativo. Si se coloca una átomo en un campo magnético externo potente, el momento magnético de su núcleo tiende a alinearse con (en paralelo) o contra (en sentido antiparalelo) el campo externo \cite{Canals_2008}.

Durante la década de 1930 Isidor Isaac Rabi, profesor de la Universidad de Columbia, junto con su equipo de investigación utilizaba una técnica denominada resonancia de haces moleculares para estudiar las propiedades magnéticas de los átomos y las moléculas. En 1939, y por consejo  del físico holandés Cornelius J. Gorter,  quien había intentado medir la resonancia magnética de núcleos de Li en LiCl sólido mediante técnicas microcalorimétricas sin resultados favorables, añadió ondas de radio a los haces moleculares mientras variaba la potencia del campo magnético y encontró que con el estímulo apropiado, los momentos magnéticos de los núcleos podían invertirse o cambiar su orientación en relación al campo magnético absorbiendo energía de la señal de radio. Posteriormente hizo experimentos variando la frecuencia de radio en lugar de la potencia del campo magnético, y descubrió que de esta manera se amplía el espectro de las señales resultantes y el espectro de la luz visible. Este método se convirtió en la base de la espectroscopia de radiofrecuencias, que revolucionó el análisis químico y es un componente esencial en el desarrollo de las exploraciones mediante resonancia magnética como herramienta de diagnóstico médico. El aporte que hicieron sus descubrimientos le valieron a Rabi el premio Nobel de física en 1944 \cite{Shampo2012,Canals_2008}.

Con el comienzo de la Segunda Guerra Mundial se interrumpieron las investigaciones sobre resonancia magnética nuclear \cite{Canals_2008,Luiten2003}. Al fin de ésta, en 1945, dos grupos de Estados Unidos, trabajando independientemente, intentaron  medir resonancia magnética nuclear en materia condensada, a pesar de que Gorter no lo había logrado. Uno de estos dos grupos, dirigido por E.M. Purcell, trabajaba en la Universidad de Harvard, Massachusetts. El otro, dirigido por Felix Bloch, lo hacía en la Universidad de Stanford, California. Ambos investigadores no se conocían \cite{Luiten2003}. La primera detección de resonancia magnética nuclear en materia condensada fue conseguida por el equipo de Purcell, Torrey y Pound en Harvard. El experimento consistió en situar un bloque de 850 $cm^3$ de parafina sólida, en una cavidad resonante de radiofrecuencia, alimentada a 30 MHz por medio de un circuito. Este conjunto se hallaba dentro de un electroimán de campo magnético variable. La señal de salida del resonador se balanceó en fase y amplitud con la porción de la señal de salida, y al encender el campo magnético se observó una aguda absorción por resonancia para un valor del campo de 7100 Gauss \cite{Purcell1946}. De manera completamente independiente, el equipo de Felix Bloch en Stanford consiguió, a fines 1945, detectar también señales de RMN de protón, utilizando un equipo experimental muy diferente y sobre una muestra de agua. En lugar de un único circuito de radiofrecuencia, en el experimento de Bloch se utilizaron dos bobinas mutuamente ortogonales, una para la excitación y la otra como detector; las dos bobinas se hallaban dentro de un fuerte campo magnético variable, perpendicular a ambas bobinas, con este diseño buscaba detectar sobre la segunda bobina la fuerza electromotriz inducida por la precesión de la magnetización nuclear de la muestra, provocada por la primera bobina al producirse la absorción \cite{Bloch1946}. En 1952, Bloch y Purcell compartieron el premio Nobel de física por estos experimentos \cite{Detre2007}.

El siguiente avance se le debe a E. L. Hahn, quien en 1949 quien siguió la idea de Bloch de producir una corta excitación mediante un pulso de radiofrecuencia, y descubrió un fenómeno conocido como Free Induction Decay (FID), base de las secuencias usadas actualmente y de  de gran importancia para la medición de los tiempos de relajación. En un principio, Hahn atribuyó estas señales aparentemente falsas a un fallo en su equipo electrónico. Tras un estudio más profundo, reconoció que estaban causadas por la aceleración y desaceleración de los núcleos giratorios debido a las variaciones en los campos magnéticos locales. Al aplicar dos o tres impulsos de radio cortos y, a continuación, escuchar el eco, Hahn descubrió que podía obtener información aún más detallada sobre la relajación del espín nuclear de lo que era posible con un único impulso \cite{Canals_2008}.

A finales de la década de 1950, Russell Varian propuso un nuevo método de impulsos denominado resonancia magnética nuclear con transformada de Fourier. Prácticamente al mismo tiempo, Irving Lowe y Richard E. Norberg, de la Universidad de Washington en St. Louis, demostraron experimental y teóricamente cómo era posible obtener todos los resultados disponibles de los experimentos con onda continua mediante la manipulación matemática de las señales producidas en un experimento con impulsos. Sin embargo, en aquel momento el proceso matemático necesario para analizar los datos de los impulsos (transformación de Fourier) no resultaba práctico debido a las limitaciones de los equipos informáticos de la época.

En 1966 Richard R Ernst y Wess W Anderson  aplican la transformada de Fourier a la espectroscopía por resonacia magnética. Utilizando la FID de Hahn y analizando la transformada de la respuesta del sistema, aumentaron la razón señal/ruido; esta técnica abre las puertas al análisis computacional de las señales, reduciendo significativamente el tiempo de registro.  Para entonces, los avances realizados en el campo de la informática hacían que la transformada de Fourier resultara práctica. Hoy es posible emplear la resonancia magnética nuclear para analizar muestras muy pequeñas de un material o identificar átomos poco comunes en muestras más grandes. En 1991, Ernst obtuvo el premio Nobel de química por sus contribuciones al desarrollo de la espectroscopia de la resonancia magnética nuclear de alta resolución. 

Hasta ese momento, en los avances realizados en la resonancia magnética no se había pensado en utilizarla para obtener imágenes, pero comenzaban a verse aplicaciones clínicas. En 1969, Raymond Damadian, un médico del Downstate Medical Center de Brooklyn (Nueva York), comenzó a idear la forma de utilizar esta técnica para detectar los primeros signos del cáncer en el organismo. En un experimento realizado en 1970, Damadian extirpó una serie de tumores de rápido crecimiento que se habían implantado en ratas de laboratorio y comprobó que la resonancia magnética nuclear de los tumores era diferente de la de los tejidos normales. En 1971, Damadian publicó los resultados de sus experimentos en la revista Science. Sin embargo, aún no se había demostrado la fiabilidad clínica del método de Damadian en la detección o diagnóstico del cáncer \cite{Canals_2008}.

El gran avance técnico que hizo posible producir una imagen útil a partir de las señales de resonancia magnética nuclear de tejidos vivos lo realizó el químico estadounidense Paul Lauterbur en 1971, cuando observó al químico Leon Saryan repetir los experimentos de Damadian con tumores y tejidos sanos de ratas. Lauterbur llegó a la conclusión de que la técnica no ofrecía la información suficiente para diagnosticar tumores y se propuso idear un método práctico para obtener imágenes a partir de la resonancia magnética nuclear. La clave estaba en ser capaz de localizar la ubicación exacta de una determinada señal de resonancia magnética nuclear en una muestra: si se determinaba la ubicación de todas las señales, sería posible elaborar un mapa de toda la muestra \cite{Luiten2003,Jara2013}. La idea de Lauterbur consistió en superponer al campo magnético estático espacialmente uniforme un segundo campo magnético más débil que variara de posición de forma controlada, creando lo que se conoce como gradiente de campo magnético. En un extremo de la muestra, la potencia del campo magnético sería mayor, potencia que se iría debilitando con una calibración precisa a medida que se fuera acercando al otro extremo. Dado que la frecuencia de resonancia de los núcleos en un campo magnético externo es proporcional a la fuerza del campo, las distintas partes de la muestra tendrían distintas frecuencias de resonancia. Por lo tanto, una frecuencia de resonancia determinada podría asociarse a una posición concreta. Además, la fuerza de la señal de resonancia en cada frecuencia indicaría el tamaño relativo de los volúmenes que contienen los núcleos en distintas frecuencias y, por tanto, en la posición correspondiente. Las sutiles variaciones de las señales se podrían utilizar entonces para representar las posiciones de las moléculas y crear una imagen \cite{Andrew2007}. Por su parte, en 1976 Peter Mansfield, de la Universidad de Nottingham, Inglaterra, desarrolló una técnica ultrarrápida para obtener imágenes con resonancia magnética conocida como técnica ecoplanar, que es la clave para crear imágenes con resonancia magnética de forma rápida \cite{Luiten2003,Andrew2007}. 

A principios de la década de 1980, la gran oleada de investigaciones relacionadas con la obtención de imágenes por resonancia magnética dieron lugar a un floreciente sector comercial. Los avances en el campo de la informática de alta velocidad y los imanes superconductores permitieron a los investigadores diseñar máquinas de resonancia magnética de mayores dimensiones con una sensibilidad y una resolución inmensamente mejores \cite{Jara2013}.

El gran avance que condujo a la resonancia magnética funcional se produjo a principios de la década de 1980, cuando George Radda y sus colegas de la Universidad de Oxford, Inglaterra, descubrieron que con la resonancia magnética se podían obtener imágenes con contraste dependiente del nivel de oxígeno de la sangre y esto podía servir para realizar un seguimiento de la actividad fisiológica,  a estó se denominó BOLD (blood oxygen level dependent) \cite{Maestu_2008}. En 1990, Seiji Ogawa informó que en estudios realizados con animales, la hemoglobina desoxigenada colocada en un campo magnético aumentaba la potencia de dicho campo, mientras que la hemoglobina oxigenada no. Ogawa demostró en estudios con animales que una zona que contiene gran cantidad de hemoglobina desoxigenada deforma ligeramente el campo magnético que rodea al vaso sanguíneo, deformación que se ve reflejada en una imagen por resonancia magnética.

Otros investigadores comenzaron a estudiar estos efectos en seres humanos. Actualmente, las imágenes obtenidas por resonancia magnética funcional se utilizan, entre otras cosas, para guiar a los cirujanos de forma que no se dañen zonas esenciales del cerebro, para detectar síntomas de infartos cerebrales y para esclarecer el funcionamiento del cerebro \cite{Maestu_2008}.

